%%%%%%%%%%%%%%%%%%%%%%%%%%%%%%%%%%%%%%%%%%%%%%%%%%%%%%%%%%%%%%%%%%%%%%%%%%%%%%%
%% Quantum FHE %%%%%%%%%%%%%%%%%%%%%%%%%%%%%%%%%%%%%%%%%%%%%%%%%%%%%%%%%%%%%%%%

\sectionframe{Quantum Fully Homomorphic Encryption}

\begin{frame}
\frametitle{Goal}
  Recall our goal: an encryption scheme with the following properties: \pause
    \begin{enumerate}
    \item the ciphertext does not reveal information about the plaintext (\textbf{security}) \pause
    \item computation can be performed on the plaintext by a party who can only access the ciphertext (\textbf{homomorphism}) \pause
    \item decryption takes roughly the same amount of time no matter which computation was performed (\textbf{compactness}) 
   \end{enumerate}
\end{frame}

\begin{frame}
\frametitle{CL scheme}
\begin{itemize} 
  \item A compact, quantum encryption scheme which is homomorphic for all Clifford circuits. \pause
  \item Unfortunately, the Clifford circuits are not universal, so this scheme is not fully homomorphic
\end{itemize}
\end{frame}
 
\begin{frame}
\frametitle{CL scheme}
The definition of Clifford circuits is closely tied to Pauli operators: 
\[ \forall C \in Clifford \text{ and } \forall Q \in Pauli, \; \exists Q' \in Pauli \; \text{s.t.  } CQ = Q'C. 
\]
\pause
Then for any quantum state \ket{\psi},  $C(Q\ket{\psi}) = Q'(C\ket{\psi})$, \\ \pause
\vspace*{5mm}
This means that we can apply any $C$ to our encrypted state $Q\ket{\psi}$ and decrypt using $Q'$ to get $C\ket{\psi}$\\ \pause
\vspace*{5mm}
How do we find $Q'$?
\end{frame}

\begin{frame}
\frametitle{CL scheme}
How do we find $Q'$? \pause
\vspace*{5mm}
$Q'$ depends both on $Q$ and on $C$.
\begin{itemize}
  \item Recall $Q = X^aZ^b$ for some $a,b \in \{0,1\}$ \pause
  \item Based on the definition of a Clifford circuit, $\forall C \in Clifford$, \[ CX^aZ^b = X^{a'}Z^{b'}C \text{ for some } a',b' \in \{0,1\} \] \pause
  \item If we learn $a',b'$ then $Q' = X^{a'}Z^{b'}$ \pause
  \item \textbf{Claim:} $\exists f_C$ such that $f_C(a, b) = (a', b')$ \pause
  \begin{itemize}
     \item Ex: Since $PX^aZ^b = X^aZ^{a \oplus b}P$, \hspace*{5mm} $f_P(a,b) = (a, a \oplus b)$ \pause  
     \item There are similar rules for the other Clifford circuits
  \end{itemize}  
\end{itemize}
\end{frame}
 
\begin{frame}
\frametitle{CL scheme}
$\exists f_C$ such that $f_C(a, b) = (a', b')$ \\ \pause
Alice sends Bob the ciphertext $Q\ket{\psi}$. Bob can simply:
\begin{enumerate}
  \item apply any $C \in Clifford$ to the ciphertext $Q\ket{\psi}$, 
  \item compute $f_C(a,b)$
  \item send $C(Q\ket{\psi})$ and $(a',b')$ to Alice for decryption (which will yield $C\ket{\psi}$ as desired). 
\end{enumerate} \pause
But Bob doesn't know $a,b$ since they form the secret key! \pause
\vspace*{5mm}

\textbf{Solution:} Since $a,b \in \{0,1\}$, Alice can use a \textit{classical} fully homomorphic encryption scheme to encrypt $(a,b) \rightarrow (c,d)$. She then sends $Q\ket{\psi}$ and $(c,d)$ to Bob who can \textit{homomorphically} evaluate $f_C(c,d)$ in step 2 without ever knowing the secret key. 
\end{frame}
 
\begin{frame}
 \frametitle{QFHE}
 \begin{itemize}
   \item Since Clifford circuits are not universal, this scheme does not achieve fully homomorphic encryption
   \item If we add T = $\begin{pmatrix} 1 & 0 \\ 0 & e^{\im \pi / 4}  \end{pmatrix}$ to the Clifford circuits, we get a universal gate set
   \item But the T-gate introduces non-linear error: \\ 
   \[  TX^aZ^b = X^aZ^{a \oplus b}P^aT  \]
 \end{itemize}
\end{frame}
 
\begin{frame}
 \frametitle{Error Correction}
 \begin{itemize}
    \item We need to solve how to update the keys when applying a T-gate to achieve fully homomorphic encryption
   \item Goal: apply $P^{-1}$ iff $a = 1$. 
   \item We want Bob to be able to do this but he doesn't know $a$ since it is part of the secret key
 \end{itemize}
\end{frame}
 
\begin{frame}
\frametitle{EPR scheme}
\begin{enumerate}
   \item Alice sends Bob $Q\ket{\psi}$ and Bob performs computations on this ciphertext
   \item When Bob applies a T-gate, he uses an auxiliary qubit to entangle the conditional $P$ error correction. The variable $k_i$ is introduced for the $i$-th T-gate applied. 
   \item Alice receives back both the ciphertext (modified by the computations Bob has performed) and all of these entangled qubits. She must perform some operations and measurements on each $k_i$ to determine whether or not to correct the $P$ error. 
   \item Alice corrects the $P$ error and then decrypts as usual
\end{enumerate}
\end{frame}

\begin{frame}
\frametitle{EPR scheme - limitations}
\begin{itemize}
  \item Since $P$ error correction is delayed until decryption, the amount of work Alice has to do is no longer independent of the computations Bob performs
  \item This scheme is not compact if a large number of T-gate operations are performed
  \item Specifically, the complexity of decryption is $O(R^2)$ if $R$ is the number of T-gates applied
  \end{itemize}
\end{frame}
 
 
\begin{frame}
\frametitle{AUX scheme}
\begin{enumerate}
   \setcounter{enumi}{-1}
   \item During key generation, AUX outputs a number of auxiliary states in the evaluation key
   \item Alice sends Bob $Q\ket{\psi}$ and Bob performs computations on this ciphertext
   \item When Bob applies a T-gate, he combines some of the auxiliary states to automatically correct the error
   \item Alice receives back the ciphertext (modified by the computations Bob has performed) 
   \item Alice decrypts as usual
\end{enumerate}
\end{frame}
 
\begin{frame}
\frametitle{AUX scheme - limitations}
\begin{itemize}
  \item Combining auxiliary states is expensive since it introduces ``cross terms'' which need to be corrected when T-gates are applied in the future
  \item This scheme is only efficient for a \textit{constant} number of T-gates since the evaluation key size grows with power exponential in the number of T-gates 
  \item AUX must be a symmetric-key encryption scheme since generating the auxiliary qubits depends on the encryption key (and Bob needs the key to combine states?)
\end{itemize}
\end{frame}
 
\begin{frame}
 \frametitle{Dulek et al. scheme}
 \begin{enumerate}
   \item Alice prepares error correction gargets and sends Bob $Q\ket{\psi}$ along with the gadgets. Bob performs computations on this ciphertext
   \item When Bob applies a T-gate, he teleports the ciphertext through the gadget to correct that error
   \item Alice receives back the ciphertext (modified by the computations Bob has performed)
   \item Alice decrypts as usual
\end{enumerate}
\end{frame}
 
\begin{frame}
\frametitle{Dulek et al. scheme - limitations}
\begin{itemize}
  \item Alice needs to prepare a gadget for each T-gate (gadgets are destroyed after use)
  \item Bob cannot create the gadgets otherwise the security of the scheme would break 
  \item For this scheme to be efficient, there can only be a polynomial number of T-gates (so that only a polynomial number of gadgets need to be prepared). This is called ``levelled FHE'' 
\end{itemize}
\end{frame}
 
\begin{frame}
\frametitle{Conclusions}
\begin{itemize}
  \item ``Historically, such results appeared as precursors to the breakthrough result establishing classical fully homomorphic encryption" - Broadbent, Jeffery 2015
  \end{itemize}
  \end{frame}
 
 \begin{frame}
 \frametitle{Dulek et al}
 \begin{itemize}
   \item \textbf{Key Gen:} 
   \begin{itemize}
     \item Generate classical keys
     \item Prepare gadgets
   \end{itemize}
   \item \textbf{Encryption:} 
   \begin{itemize}
     \item quantum one-time pad to message
     \item classical FHE on secret keys
   \end{itemize}
   \item \textbf{Evaluation:}
    \begin{itemize}
     \item for each Clifford circuit, classically update keys (as in CL scheme)
     \item for each T-gate, update keys using a gadget
   \end{itemize}
   \item \textbf{Decryption:}
   \begin{itemize}
     \item decrypt pad keys using classical FHE decryption
     \item use pad keys to decrypt one-time pad
    \end{itemize}
 \end{itemize}
 \end{frame}
