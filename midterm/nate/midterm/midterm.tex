\documentclass{article}

\usepackage{natbib}
\usepackage{amsmath}
\usepackage{paralist}

\newcommand{\ket}[1]{| #1 \rangle}

\title{Alternative QOTP schemes for QHE}
\author{Ben Hamlin, Nate Launchbury, and Stephen Libby}
\date{}

\begin{document}
\maketitle

Quantum computing is a promising new avenue for algorithm development.  However,
quantum computers will not be available to the general public for the
foreseeable future.  One potential solution allows users to access a quantum
server to run their computations in a “quantum cloud”, but this comes with a
cost.  To run a computation on a server, the user must be willing to share their
data.  This loss of privacy is unacceptable for many users.

A solution to the corresponding problem in the classical setting has recently
been proposed in the form of fully homomorphic encryption.  The idea comes from
a simple question: can we encrypt data and run arbitrary computations on it,
without ever decrypting the data?  Surprisingly, Rivest et al. showed that this
was possible \cite{rivest}.  If we want to perform computations in the quantum
cloud, we would like the privacy provided by fully homomorphic encryption but
the ability to pass around quantum data. This is the premise of quantum fully
homomorphic encryption (QHE).

\citet{liang2015}: This paper presents two QHE scheme based on CSS codes. One is
symmetric, and the other is asymmetric. The encryption key is randomly chosen
from the set of CSS codes, they encrypt simply by applying the code, and decrypt
by reversing the CSS code. The evaluation of $H$ and $CNOT$ gates are simple.
Since their encryption is just a CSS code, the fault tolerant versions of $H$
and $CNOT$ are already homomorphic. For the $T$ gate they need an extra state
$\ket{Phi}$. Since this state needs the same code that was applied when
encrypting, the state has to be encrypted along with the message. They extend
this to the asymmetric setting, by randomly adding errors to the encrypted code.
It is still possible to apply the fault tolerant gates as before, but the
encrypted message has to be sent back and re-encrypted periodically.

\citet{lai2017}: Lai and Chung show that an IT-secure symmetric-key QFHE scheme
cannot exist. They then give an IT-secure QHE scheme which can evaluate all
circuits in the enlarged instantaneous quantum polynomial-time circuits called
$IQP^+$, which is neither trivial nor universal. IQP+ is defined by the quantum
gates X, Y, CNOT, R($\theta$), and CR($\theta$). R($\theta$) is a rotation by
$\theta$ degrees and CR($\theta$) is a controlled version of the R($\theta$)
gate. These two gates in particular can be implemented using quantum stabilizer
codes which are simply quantum error-correcting codes where ancilla qubits are
used to protect the qubits containing data. Indeed, concatenating these
stabilizer codes together is the basis of the QHE scheme which Lai and Chung
give which can be evaluated for any circuit in $IQP^+$. Of course, the main
limitation of this scheme is that it is not universal since $IQP^+$ does not
contain a universal set of gates. Since the scheme does have IT-security, it
follows from the first result in this paper that it cannot support full quantum
computation.

Another class of QHE schemes has already been proposed. \citet{broadbent2015}
and \citet{dulek2016} both use a quantum analog to Vernam’s one-time pad,
formulated in \citet{ambainis2000}. Like Vernam’s cipher, the quantum one-time
pad (qOTP) is information-theoretically (IT) secure. It involves applying to
each qubit a random Pauli operation, $X^aZ^b$ for $a,b \in \{0,1\}$.
\citet{ambainis2000} show that two random bits per qubit (corresponding to $a$
and $b$) are both necessary and sufficient to produce IT-secure encryption.

\citet{ambainis2000}: The authors define a private quantum channel (PQC), a set
of probabilistically applied unitary operators or \textit{superoperator}, after
whose application to a state, the resulting density matrix provides no
information about the original state. They prove that applying random Pauli
matrices to a state implements a PQC, and that $2n$ bits of randomness are
enough to implement an $n$-qubit PQC. Furthermore, they show that their scheme
reduces to classical OTP and requires just $n$ bits of randomness when applied
to classical states. Hence their scheme can be viewed as the natural quantum
extension of OTP (QOTP).

\citet{liang2014}: The author proposes a variety of specific schemes that
implement QOTP as defined in \cite{ambainis2000}. These schemes involve applying
Pauli rotations about the $x$-, $y$-, and $z$-axes, determined by a shared
secret key. The QFHE scheme the author proposes is both efficient and
information-theoretically secure (ITS). This scheme is of dubious usefulness,
however: no decryption is required to evaluate a circuit on the ciphertext, so
the scheme is technically QFHE by the author's definition, but the evaluator
must have access to the secret key, which violates other authors' definitions.
Of more practical use, the author provides an ITS-QHE scheme that is homomorphic
over Clifford gates and does not require the evaluator to have access to the
secret key.

\citet{broadbent2015}: This paper defines what it means for a quantum scheme to
be CPA and CPA-mult secure. It then goes on to define the concept of
homomorphic, and fully homomorphic, encryption in a quantum setting. Finally it
gives homomorphic encryption schemes. CL, a quantum one time pad that's
homomorphic on Clifford gates. EPR, a scheme based on adding EPR pairs to help
correct for the errors introduced by adding T gates. AUX, a scheme designed to
overcome the decryption increase in EPR by supplying auxiliary qubits to correct
for the errors immediately. In the CL scheme they point out that  Pauli gates
are a normal subgroup of the Clifford group, so if C is a clifford gate, then
$CQ = Q'C$ for some $Q,Q'$ in the Pauli gates. Suppose we have a QOTP
$X^{a_1}Z^{b_1}\otimes \ldots X^{a_n}Z^{b_n}$, then $C X^{a_1}Z^{b_1}\otimes
\ldots X^{a_n}Z^{b_n} = X^{a_1'}Z^{b_1'}\otimes \ldots X^{a_n'}Z^{b_n'}C$ For
some new key. This means applying a circuit $C$ is the same as updating the key
correctly. EPR scheme: Trying this same trick with a T gate doesn't work,
because $TX^aZ^b = X^aZ^{a\oplus b}P^aT$. The extra factor of $P^a$ forbids them
from modifing the key $(a,b)$. They get around this by providing a pair of
entangled qubits $\ket{\Phi^+}$. They modify one of the qubits by computing
$\text{CNOT}(T\otimes I)$, tying it's state to the encrypted qubit. Then they
apply $HP^{f_t}$ to the other and measure. Here $f_t$ count's the number of $T$
gates we've applied. This corrects for the extra factor of $P$. However this
means that the $HP$ gates are applied at decryption, which adds to the
decryption time. Specifically it add $O(n^2)$ $P$ gates for $n$ $T$ gates. AUX
scheme: The EPR scheme allows us to have $T$ gates, but it adds more to the
decryption algorithm. This can quickly break the compactness restriction. An
attempt to get around this is to encode some ``auxiliary states'' to account for
the $P$ terms introduced. This has two major consequences. Since we have to
prepare the auxiliary qubits in advance, and they depend on the encryption key,
this has to be symmetric key encryption. The other consequence is that the cost
of correcting the $P$ terms grows by adding cross terms that need to be
corrected, similar to multiplying polynomials. This means that the evaluation
key grows exponentially, since it has to account for all possible $P$
corrections. The result is that this scheme can only have a constant number of
$T$ gates.

\citet{dulek2016}: The authors present a new scheme for QHE which is both
compact and allows for efficient evaluation of arbitrary polynomial-sized
quantum circuits based on the CL scheme of \cite{broadbent2015}. The error that
a $T$ gate may incur is based on part of the secret key and so cannot be
corrected during evaluation directly by a party for whom the secret key is
unknown. To solve this, the authors propose so-called “gadgets” which consist of
EPR pairs prepared in a way dependent on this secret key. Upon evaluating a $T$
gate, the evaluator teleports the encrypted qubit through the gadget based on
the encryption of the secret key. The gadgets are set up in such a way that this
process completely corrects any unwanted phase-error introduced by the $T$ gate.
By using gadgets to correct error during the evaluation of $T$ gates, this
scheme has the advantages of not accumulating error during evaluation and
requires only a single gadget per $T$ gate. It is also based on any classical
leveled FHE scheme without making any assumptions about how this scheme works.
This implies that any efficiency improvements in classical FHE will result in
proportional efficiency improvements for this scheme.

qOTP shares another property with Vernam’s cipher: It is “somewhat” homomorphic.
Specifically, if we define $CL$, the Clifford group, as the set of operations
generated by $\{X,Z,H,P,CNOT\}$, then for any encryption $\ket{\psi} =
X^aZ^b\ket{\phi}$ and any gate $C \in CL$, $\exists c,d \in \{0,1\}$ s.t.
$C\ket{\psi} = X^cZ^d\ket{\phi}$. Although this makes it possible to perform
computations on ciphertexts, it is still just \textit{somewhat} homomorphic
because $CL$ is not universal. Specifically, the Toffoli gate (and the $T$ gate,
from which it can be constructed) is non-Clifford, meaning that $CL$ is not even
universal for classical reversible computation.

This result is strengthened by \citet{yu2014}. They show that the ciphertext of
any IT-secure QHE must increase exponentially with the number of operations, as
long as the set of permissible gates $S$ is universal for classical reversible
computation or is a universal approximator for quantum computation.
Exponentially increasing ciphertext breaks the locality requirement that
Broadbent and Jeffery put on QHE---that decryption complexity must be
independent of the number of computations. This means that any QHE primitive
(e.g. qOTP) must relax at least one of these constraints: IT-security,
universality, or locality. \citet{broadbent2015} and \citet{dulek2016} both
relax the constraint that the scheme must be IT-secure, relying instead on
computational security. Nonetheless, they use IT-secure qOTP as a primitive.

\citet{alagic2016}: The authors define quantum analogues to various important
classical security definitions, including certain proofs of equivalence. In
particular, they give the definition of a quantum one-time pad and both
private-key and public-key encryption in the quantum setting. Additionally, they
provide definitions of the security properties of indistinguishability and
semantic security along with a proof that the two definitions are equivalent.
Finally, they show that quantum-secure one-way functions are enough to build a
private-key quantum encryption scheme which is secure under chosen-ciphertext
attack and that quantum-secure trapdoor one-way permutations imply
semantically-secure public-key quantum encryption. Of particular interest is the
definition of indistinguishability under chosen-plaintext attack (IND-CPA)
security. Under this definition, an adversary is allowed to output a challenge
plaintext along with some auxilliary information which even allows entanglement
of the output with some internal state belonging to the adversary. The
adversary’s goal is to distinguish between the encryption of their challenge and
the zero state, where IND-CPA security implies that they can do no better than
guessing.

\citet{yu2014}: Since quantum computing has provided information-theoretic
security (ITS) in other domains, we might hope that it would be possible to
implement quantum homomorphic encryption with ITS. Alternatively, even if
ITS-QHE is impossible, we might hope that ITS classical fully-homomorphic
encryption (FHE) would be possible over quantum channels. On the contrary, Yu et
al. prove a ``no-go'' theorem: They show that, in any QHE scheme where Bob
obtains no information about Alice's data while performing his computation, the
size of the ciphertext must increase exponentially after each operation,
precluding any efficient ITS-QFHE scheme. This theorem holds even if the set of
gates Bob can compute is not universal over unitary gates, but merely over
classical reversible gates.

\paragraph{Our question (revised)}: \citet{liang2014} define several kinds of
QOTP, all of which support only a limited set of gates to be applied without
access to the secret key. Can we find a QOTP scheme that supports a universal
set of gates $G$?

In other words, is there some sequence of gates $S$ with $|S| = s$ such that for
any quantum state $\ket\phi$ and operation $U$, in some universal set of gates
$G$, and bits $b_1,\dots,b_s \in \{0,1\}$,
\begin{align}
U (\prod_{i=1}^s S_i^{b_i}) \ket\phi = (\prod_{i=1}^s S_i^{b’_i}) U \ket\phi
\end{align}
where $b’_1,\dots,b’_s \in \{0,1\}$ and $\phi : (b_1,\dots,b_s, U) \mapsto
b’_1,\dots,b’_s$ is a known mapping? One way to achieve this would be to let $N
= \{(\prod_{i=1}^s S_i^{b_i})\}$ be a normal subgroup of $G$, since then
\begin{align}
U (\prod_{i=1}^s S_i^{b_i}) = (\prod_{i=1}^s S_i^{b’_i}) U
\end{align}
With the additional constraint that $b’_1,\dots,b’_s \in \{0,1\}$ be efficiently
computable from $U$, this would give us the property that we need.

\citet{yu2014} prove that, if such a gate set exists, it cannot perfectly hide
the message unless the ciphertext grows exponentially with each homomorphic
operation applied to it. Hence, we cannot hope that a scheme such as the one
defined above would be IT-secure. Instead, we hope to apply the definition of
quantum independence under chosen-plaintext attack (IND-CPA) provided by
\citet{alagic2016} to show that such a scheme can be computationally secure.

If we cannot find a QOTP scheme that supports a universal set of gates, we have
two ancillary questions:
\begin{compactenum}
\item
Is it impossible for an IND-CPA QOTP scheme to support a universal set of gates?
Can we prove it?
\item
Can we find an IND-CPA QOTP scheme that supports at least the Toffoli gate?
\end{compactenum}
Answering the latter question in the affirmative would allow us to construct a
QHE scheme that supports classical reversible computation, which is also
impossible to do efficiently in an IT-secure setting.

\bibliographystyle{plainnat}
\bibliography{midterm}

\end{document}
