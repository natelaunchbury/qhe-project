%%%%%%%%%%%%%%%%%%%%%%%%%%%%%%%%%%%%%%%%%%%%%%%%%%%%%%%%%%%%%%%%%%%%%%%%%%%%%%%
%% Classical FHE %%%%%%%%%%%%%%%%%%%%%%%%%%%%%%%%%%%%%%%%%%%%%%%%%%%%%%%%%%%%%%

\sectionframe{Classical Fully Homomorphic Encryption}

\begin{frame}
\frametitle{Homomorphisms}
\textbf{Homomorphism:} structure preserving map
\pause
\begin{itemize}
  \item determinent: $|AB| = |A|\cdot |B|$
\pause
  \item $e^{x+y} = e^x\cdot e^y$
\pause
  \item $length (a\pp b) = length(a) + length(b)$
\pause
\end{itemize}
$\ $\\
This idea is in Algebra, Topology, Logic, Linear Algebra $\ldots$
\end{frame}

\begin{frame}
\frametitle{Homomorphic Encryption}
If a homomorphism is a structure preserving,\\
what structure are we preserving?\\
\pause
$\ $\\
$\ $\\
correctness property for encryption\\
\pause
$\ $\\
If we encrypt something, then decrypt,\\
we should get the original message.\\
\pause
$\ $\\
$Dec_{sk} \circ Enc_{pk} = id$\\
\end{frame}

\begin{frame}
\frametitle{Homomorphic Encryption}
can we run a computation on the encrypted text?\\
\pause
We encrypt, run our computation $f$, and decrypt.\\
What structure is preserved?\\
$\ $\\
\pause
$Dec_{sk} \circ Eval_{evk}(f) \circ Enc_{pk} = f$\\
$\ $\\
$\ $\\
\pause
In reality this is too strong, we instead say\\
$\ $\\
$P[Dec_{sk} (Eval_{evk}(f, Enc_{pk}(m))) \ne m] \le \eta(N)$\\
$\ \ \text{where}\ \eta$ is negligable.
\end{frame}

\begin{frame}
\frametitle{Trivial Homomorphic Encryption}
There's a very simple solution to this.\\
\pause
let $Enc$ and $Dec$ be a secure encryption scheme.\\
$\ $\\
\pause
$Enc_k'(m) = Enc_k(m)$ \\
\pause
$Eval_k(f,c) = c \pp "\$f"$ \\
\pause
$Dec_k'(c\$f) = f(Dec_k(c))$ \\
$\ $\\
\pause
This is the \textit{trivial homomorphic encryption scheme}\\

\end{frame}

\begin{frame}
\frametitle{Trivial Homomorphic Encryption}

This scheme is correct\\
$\ $\\
\pause
$Dec_{ks} \circ Eval_{evk}(f) \circ Enc_{pk} = f \circ Dec_{sk} \circ Enc_{pk} = f$\\
$\ $\\
\pause
But this isn't what we meant\\
$\ $\\
\pause
To avoid this we'll add a new requirement\\
The size of the ciphertext must not depend on the function.\\
Compactness: $|Eval_k(f,c)| = Poly(N)$\\
$\ $\\
\pause
side effect: we can only encrypt text up to length $Poly(N)$\\

\end{frame}

\begin{frame}
\frametitle{Representation}
We represent our functions as circuits\\
$\ $\\
\pause
\textit{depth}: maxmimum length from input to output of a circuit\\
$\ $\\
\pause
$Eval_{evk}(f)$ does not have to be a new circuit\\
it is only true that $Eval_{evk}(f,c)$ is a valid ciphertext\\
$\ $\\
\pause
Furthermore $Dec_{sk}\circ Enc_{pk} = id$ is no longer required.\\
only $Dec_{sk}\circ Eval_{evk}(f) \circ Enc_{pk} = f$\\
$\ $\\
This is because $Dec_{sk}\circ Eval_{evk}(id) \circ Enc_{pk} = id$\\
\end{frame}

\begin{frame}
\frametitle{Homomorphic Encryption}

There are many ways to fulfill correctness and compactness.
\begin{itemize}
\pause
  \item Somewhat Homomorphic: Correct, but not compact
\pause
  \item $\mathcal{F}$-Homomorphic: Correct and compact for circuits in $\mathcal{F}$
\pause
  \item Levelled Homomorphic: $\mathcal{F}$-Homomorphic for limited depth
\pause
  \item Fully Homomorphic: Correct and compact for all circuits
\pause
  \item i-hop Homomorphic: Correct and compact $i$ calls to $Eval_{evk}$
\end{itemize}
$\ $\\
\pause
Our goal is Fully Homomorphic Encryption
\end{frame}

\begin{frame}
\frametitle{Implementation}
Most implementations of this come from lattice problems\\
$\ $\\
We don't have time to go over the necessary theory here.\\
$\ $\\
So we'll give an idea\\
\end{frame}

\begin{frame}
\frametitle{Implementation}
An HE scheme is \textit{bootstrapable} if it can homomorphically evaluate
$Dec_{sk}$ along with at least one NAND gate.\\
$\ $\\
\pause
$Enc_{pk}' = Enc_{pk} \circ Enc_{pk}$\\
$\ $\\
\pause
$Eval_{evk}(f)' = Eval_k \circ \ldots \circ Eval_0$\\
$\ \ \text{where}\ (N_1\ldots N_k) = NANDS(f)$\\
$\ \ \ \ \ \ \ Eval_i = Eval_{evk}(Enc_{pk}\circ N_i\circ Dec_{sk})$\\
$\ $\\
\pause
$Dec_{sk}' = Dec_{sk} \circ Dec_{sk}$\\
$\ $\\
\pause
We can't evaluate $Dec_{sk}$ because then we'd need to know $sk$\\
\pause
but we can encrypt $sk$ since we can evaluate $Dec$\\
\pause
This process is called \textit{recryption}\\
\end{frame}
