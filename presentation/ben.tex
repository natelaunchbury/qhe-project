%%%%%%%%%%%%%%%%%%%%%%%%%%%%%%%%%%%%%%%%%%%%%%%%%%%%%%%%%%%%%%%%%%%%%%%%%%%%%%%
%% QOTP %%%%%%%%%%%%%%%%%%%%%%%%%%%%%%%%%%%%%%%%%%%%%%%%%%%%%%%%%%%%%%%%%%%%%%%

\sectionframe{A Homomorphic Primitive: \\ Quantum One-time Pad}

\begin{frame}
\frametitle{Classical One-time Pad}
\vspace{7mm}
\begin{itemize}
\item \textbf{Con:} Large, single-use key (as many bits as message)
\item \textbf{Pro:} Arbitrary bit flips (homomorphism)
\end{itemize}
\vspace{7mm}
\textit{Intuition:} Perhaps a quantum extension of one-time pad would be
similarly homomorphic.
\end{frame}

\begin{frame}
\frametitle{Quantum One-time Pad}
\only<1>{
Can we extend classical one-time pad to a quantum setting?
\vspace{7mm}
\begin{itemize}
\item Method established as early as 2000
\begin{itemize}
\item CITE Ambainis et al., 2000
\end{itemize}
\item Homomorphic properties examined starting in 2014
\begin{itemize}
\item CITE Liang et al., 2014
\item CITE Broadbent and Jeffery, 2015
\end{itemize}
\end{itemize}
}
\only<2>{
Facts:
\begin{itemize}
\item There is a quantum one-time pad that extends classical counterpart
\item It has information-theoretic security
\item It is \textit{somewhat} homomorphic
\end{itemize}
}
\end{frame}

\sectionframe{From Classical to Quantum One-time Pad}

\begin{frame}
\frametitle{Classical One-time Pad}
\begin{columns}
\column{0.7\textwidth}
\only<1>{
Encryption: \(c = m \oplus k\)
Decryption: \(m = c \oplus k\)
}
\only<2>{\[\Qcircuit @C=2mm @R=3mm {
   \lstick{k} & \multigate{1}{\mathcal R} & \qw & \rstick{k}
\\ \lstick{m} & \ghost{R}        & \qw & \rstick{c}
}\]\vspace{7mm}\[(k,m) \xmapsto{R} (k, c = m \oplus k)\]}
\only<3>{\[\Qcircuit @C=2mm @R=3mm {
   \lstick{k} & \ctrl{1} & \qw & \rstick{k}
\\ \lstick{m} & \targ    & \qw & \rstick{c}
}\]}
\column{0.3\textwidth}
key:        \[k \in \{0,1\}\] \vspace{5mm} \\
message:    \[m \in \{0,1\}\] \vspace{5mm} \\
ciphertext: \[c \in \{0,1\}\]
\end{columns}
\end{frame}

\begin{frame}
\frametitle{IT-Secure Quantum One-time Pad}
\begin{columns}
\column{0.7\textwidth}
\center
\only<1>{\[\Qcircuit @C=2mm @R=3mm {
   \lstick{k}      & \control\cw & \cw & \rstick{k}
\\ \lstick{\ket m} & \targ\cwx   & \qw & \rstick{\ket c}
}\]}
\only<2>{
\[\rho = \begin{pmatrix} a&b\\c&d \end{pmatrix}, b = \bar{c}, a+d = 1\]
\begin{align*}
\frac{1}{2}(X \rho X + \rho)
        &= \frac{1}{2}\left( \begin{pmatrix} d&c\\b&a \end{pmatrix}
                + \begin{pmatrix} a&b\\c&d \end{pmatrix} \right)
\\      &= \frac{1}{2}\begin{pmatrix} 1&b+c\\b+c&1 \end{pmatrix}
\end{align*}
Not totally mixed.
}
\only<3>{\[\Qcircuit @C=2mm @R=3mm {
   \lstick{k_1}    & \control\cw & \cw          & \cw & \rstick{k_1}
\\ \lstick{k_2}    & \cw\cwx     & \control\cw  & \cw & \rstick{k_2}
\\ \lstick{\ket m} & \targ\cwx   & \gate{Z}\cwx & \qw & \rstick{\ket c}
}\]}
\only<4>{
\[\rho = \begin{pmatrix} a&b\\c&d \end{pmatrix}, b = \hat{c}, a+d = 1\]
\begin{align*}
\frac{1}{4}(\QGate{X}Z \rho ZX + X \rho X + Z \rho Z + \rho)
        = \frac{1}{2}\begin{pmatrix} 1&0\\0&1 \end{pmatrix}
\end{align*}
Totally mixed!
}
\only<5>{
Encryption: \(\rho \xmapsto{k_1,k_2} X^{k_1}Z^{k_2} \rho Z^{k_2}X^{k_1}\)
Decryption: \(\sigma \xmapsto{k_1,k_2} X^{k_1}Z^{k_2} \sigma Z^{k_2}X^{k_1}\)
}
\column{0.3\textwidth}
\only<1-2>{
key:        \[k \in \{0,1\}\] \vspace{5mm} \\
message:    \[m = \alpha\ket0 + \beta\ket1\] \vspace{5mm} \\
ciphertext: \[c = \alpha'\ket0 + \beta'\ket1\]
}
\only<3->{
key:        \[k_1,k_2 \in \{0,1\}\] \vspace{5mm} \\
message:    \[m = \alpha\ket0 + \beta\ket1\] \vspace{5mm} \\
ciphertext: \[c = \alpha'\ket0 + \beta'\ket1\]
}
\end{columns}
\end{frame}

\sectionframe{But what about homomorphism?}

\begin{frame}
\frametitle{Homomorphism}
\begin{columns}
\column{0.7\textwidth}
\only<1>{
\textit{Example:} Suppose our key is $(k_1,k_2) = (0,1)$, and we want to apply
$H$ to our encrypted message.
}
\only<2>{
\[\begin{tabular}{cc} $k_1 = 0$ & $k_2 = 1$ \end{tabular}\]
\[\sigma = X^{k_1}Z^{k_2} \rho Z^{k_2}X^{k_1} = Z \rho Z\]
}
\only<3>{
\[\begin{tabular}{cc} $k_1 = 0$ & $k_2 = 1$ \end{tabular}\]
\[\sigma' = HZ \rho ZH\]
}
\only<4>{
\[\begin{tabular}{cc} $k_1 = 0$ & $k_2 = 1$ \end{tabular}\]
\[\sigma' = XH \rho HX\]
}
\only<5>{
\[\begin{tabular}{cc} $k'_1 = 1$ & $k'_2 = 0$ \end{tabular}\]
\[\sigma' = X \rho' X\]
\textit{Notice what happened to our key}
}
\only<6>{
Does this work for all unitaries? Suppose we want to apply $Y$.
}
\only<7>{
\[\begin{tabular}{cc} $k_1 = 0$ & $k_2 = 1$ \end{tabular}\]
\[\sigma = X^{k_1}Z^{k_2} \rho Z^{k_2}X^{k_1} = Z \rho Z\]
}
\only<8>{
\[\begin{tabular}{cc} $k_1 = 0$ & $k_2 = 1$ \end{tabular}\]
\[\sigma' = YZ \rho ZY\]
}
\column{0.3\textwidth}
\[X = \begin{pmatrix} 0&1\\1&0 \end{pmatrix}\]
\vspace{3mm}
\[Y = \begin{pmatrix} 0&i\\-i&0 \end{pmatrix}\]
\vspace{3mm}
\[Z = \begin{pmatrix} 1&0\\0&-1 \end{pmatrix}\]
\vspace{3mm}
\[H = \frac{1}{\sqrt{2}}\begin{pmatrix} 1&1\\1&-1 \end{pmatrix}\]
\vspace{3mm}
Original message: $\rho$
\vspace{3mm}
Ciphertext: $\sigma$
\vspace{3mm}
Target message: $\rho' = H \rho H$
\end{columns}
\end{frame}
