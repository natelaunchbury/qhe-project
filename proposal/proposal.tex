\documentclass{article}

\usepackage[braket]{qcircuit}
\usepackage{amsmath}
\usepackage{amssymb}
\usepackage{natbib}

\title{Alternatives to quantum OTP for QHE \\
\normalsize Quantum Computing Project Proposal}
\author{Stephen Libby, Nate Launchbury, Ben Hamlin}
\date{}

\begin{document}

\maketitle

Quantum computing is a promising new avenue for algorithm development. However,
quantum computers will not be available to the general public for the
foreseeable future. One potential solution allows users to access a quantum
server to run their computations in a ``quantum cloud'', but this comes with a
cost. To run a computation on a server, the user must be willing to share their
data. This loss of privacy is unacceptable for many users.

A solution to the corresponding problem in the classical setting exists in the
form of fully homomorphic encryption. The idea comes from a simple question:
can we encrypt data and run arbitrary computations on it, without ever
decrypting the data? Surprisingly, Rivest et al. showed that this was possible
\cite{rivest1978}. If we want to perform computations in the quantum cloud, we
would like the privacy provided by fully homomorphic encryption but the ability
to pass around quantum data. This is the premise of quantum fully homomorphic
encryption (QHE).

Several QHE schemes have already been proposed. \citet{broadbent2015} and
\citet{dulek2016} both use a quantum analog to Vernam’s one-time pad, formulated
in \citet{ambainis2000}. Like Vernam’s cipher, the quantum one-time pad (qOTP)
is information-theoretically (IT) secure. It involves applying to each qubit a
random Pauli operation, $X^aZ^b$ for $a,b \in \{0,1\}$. \citet{ambainis2000}
show that two random bits per qubit (corresponding to $a$ and $b$) are both
necessary and sufficient to produce IT-secure encryption.

The qOTP primitive shares another property with Vernam’s cipher: It is
``somewhat'' homomorphic. Specifically, if we define $CL$, the Clifford group, as
the set of operations generated by $\{X,Z,H,P,CNOT\}$, then for any encryption
$\ket{\psi} = X^aZ^b\ket{\phi}$ and any gate $C \in CL$, $\exists c,d \in
\{0,1\}$ s.t. $C\ket{\psi} = X^cZ^d\ket{\phi}$. Although this makes it possible
to perform computations on ciphertexts, it is still just \textit{somewhat}
homomorphic because $CL$ is not universal. Specifically, the Toffoli gate (and
the $T$ gate, from which it can be constructed) is non-Clifford, meaning that
$CL$ is not even universal for classical reversible computation.

This result is strengthened by \citet{yu2014}, who show that any IT-secure QHE
whose set of permissible gates $S$ is universal for classical reversible
computation or is a universal approximator for quantum computation must have
ciphertext whose size increases exponentially with the number of operations.
Exponentially increasing ciphertext breaks the locality requirement that
\citet{broadbent2015} put on QHE---that decryption complexity must be
independent of the number of computations. This means that any QHE primitive
(e.g. qOTP) must relax at least one of these constraints: IT-security,
universality, or locality. \citet{broadbent2015} and \citet{dulek2016} both
relax the constraint that the scheme must be IT-secure, relying instead on
computational security. Nonetheless, they use IT-secure qOTP as a primitive.

\paragraph{Our question:} If we relax the constraint that QHE must be IT-secure,
is there some primitive that works like qOTP (where operations map one valid
encryption to another) but allows a universal set of gates to be applied without
accumulating error? If so, what is this scheme, and how many bits of randomness
per qubit would it require? If not, what makes it impossible?

According to \citet{broadbent2015}, qOTP achieves IT-security because, if $a,b$
are chosen uniformly at random, ``the encryption maps any input to the completely
mixed state (from the point of view of the adversary).'' One approach we want to
explore is to relax IT-security by stipulating instead that the random
transformation our primitive produces put our ciphertext into a state that is
\textit{computationally} indistinguishable from the completely mixed state. If
such a primitive is provably impossible without relaxing additional constraints,
this would also be an useful result.

\bibliographystyle{unsrtnat}
\bibliography{bibliography}

\end{document}
